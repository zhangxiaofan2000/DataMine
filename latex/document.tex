\section{图片}

建模中不可避免要插入图片。图片可以分为矢量图与位图。位图推荐使用 \verb|jpg,png| 这两种格式,避免使用 \verb|bmp| 这类图片,容易出现图片插入失败这样情况的发生。矢量图一般有 \verb|pdf,eps| ,推荐使用 \verb|pdf|  格式的图片,尽量不要使用 \verb|eps| 图片,理由相同。

注意图片的命名,避免使用中文来命名图片,可以用英文与数字的组合来命名图片。避免使用\verb|1,2,3| 这样顺序的图片命名方式。图片多了,自己都不清楚那张图是什么了,命名尽量让它有意义。下面是一个插图的示例代码。
\begin{figure}[!h]
	\centering
	\includegraphics[width=.6\textwidth]{smokeblk}
	\caption{电路图}
	\label{fig:circuit-diagram}
\end{figure}

注意 \verb|figure| 环境是一个浮动体环境,图片的最终位置可能会跑动。\verb|[!h]| 中的 \verb|h| 是 here 的意思, \verb|!| 表示忽略一些浮动体的严格规则。另外里面还可以加上 \verb|btp| 选项,它们分别是 bottom, top, page 的意思。只要这几个参数在花括号里面,作用是不分先后顺序的。page 在这里表示浮动页。

\verb|\label{fig:circuit-diagram}| 是一个标签,供交叉引用使用的。例如引用图片 \verb|\cref{fig:circuit-diagram}| 的实际效果是\cref{fig:circuit-diagram}。图片是自动编号的,比起手动编号,它更加高效。\verb|\cref{label}| 由 \verb|cleveref| 宏包提供,比普通的 \verb|\ref{label}| 更加自动化。 \verb|label| 要确保唯一,命名方式推荐用图片的命名方式。

图片并排的需求解决方式多种多样,下面用 \verb|minipage| 环境来展示一个简单的例子。注意,以下例子用到了 \verb|subcaption| 命令,需要加载 subcaption 宏包。

\begin{figure}
	\centering
	\begin{minipage}[c]{0.3\textwidth}
		\centering
		\includegraphics[width=0.95\textwidth]{f1}
		\subcaption{流程图}
		\label{fig:sample-figure-a}
	\end{minipage}
	\begin{minipage}[c]{0.3\textwidth}
		\centering
		\includegraphics[width=0.95\textwidth]{f1}
		\subcaption{流程图}
		\label{fig:sample-figure-b}
	\end{minipage}
	\begin{minipage}[c]{0.3\textwidth}
		\centering
		\includegraphics[width=0.95\textwidth]{f1}
		\subcaption{流程图}
		\label{fig:sample-figure-c}
	\end{minipage}
	\caption{多图并排示例}
	\label{fig:sample-figure}
\end{figure}
这相当于整体是一张大图片,大图片引用是\cref{fig:sample-figure},子图引用别分是\cref{fig:sample-figure-a}、\cref{fig:sample-figure-b}、\cref{fig:sample-figure-c}。

如果原本两张图片的高度不同,但是希望它们缩放后等高的排在同一行,参考这个例子:
\begin{figure}
	\centering
	\begin{minipage}[c]{0.48\textwidth}
		\centering
		\includegraphics[height=0.2\textheight]{cat}
		\subcaption{一只猫}
	\end{minipage}
	\begin{minipage}[c]{0.48\textwidth}
		\centering
		\includegraphics[height=0.2\textheight]{smokeblk}
		\subcaption{电路图}
	\end{minipage}
	\caption{多图并排示例}
\end{figure}

\section{绘制普通三线表格}
表格应具有三线表格式,因此常用 booktabs宏包,其标准格式如\cref{tab:001}~所示。
\begin{table}[!htbp]
	\caption{标准三线表格}\label{tab:001} \centering
	\begin{tabular}{ccccc}
		\toprule[1.5pt]
		$D$(in) & $P_u$(lbs) & $u_u$(in) & $\beta$ & $G_f$(psi.in)\\
		\midrule[1pt]
		5 & 269.8 & 0.000674 & 1.79 & 0.04089\\
		10 & 421.0 & 0.001035 & 3.59 & 0.04089\\
		20 & 640.2 & 0.001565 & 7.18 & 0.04089\\
		\bottomrule[1.5pt]
	\end{tabular}
\end{table}

其绘制表格的代码及其说明如下。
\begin{tcode}
	\begin{table}[!htbp]
		\caption[标签名]{中文标题}
		\begin{tabular}{cc...c}
			\toprule[1.5pt]
			表头第1个格   & 表头第2个格   & ... & 表头第n个格  \\
			\midrule[1pt]
			表中数据(1,1) & 表中数据(1,2) & ... & 表中数据(1,n)\\
			表中数据(2,1) & 表中数据(2,2) & ... & 表中数据(2,n)\\
			...................................................\\
			表中数据(m,1) & 表中数据(m,2) & ... & 表中数据(m,n)\\
			\bottomrule[1.5pt]
		\end{tabular}
	\end{table}
\end{tcode}

\bigskip

table环境是一个将表格嵌入文本的浮动环境。tabular环境的必选参数由每列对应一个格式字符所组成:c表示居中,l表示左对齐,r表示右对齐,其总个数应与表的列数相同。此外,\verb|@{文本}|可以出现在任意两个上述的列格式之间,其中的文本将被插入每一行的同一位置。表格的各行以\verb|\\|分隔,同一行的各列则以\&分隔。 \verb|\toprule| 、\verb|\midrule| 和 \verb|\bottomrule| 三个命令是由booktabs宏包提供的,其中 \verb|\toprule| 和 \verb|\bottomrule| 分别用来绘制表格的第一条(表格最顶部)和第三条(表格最底部)水平线, \verb|\midrule| 用来绘制第二条(表头之下)水平线,且第一条和第三条水平线的线宽为 1.5pt ,第二条水平线的线宽为 1pt 。引用方法与图片的相同。

\section{公式}

数学建模必然涉及不少数学公式的使用。下面简单介绍一个可能用得上的数学环境。

首先是行内公式,例如 $ \theta $ 是角度。行内公式使用 \verb|$  $| 包裹。

行间公式不需要编号的可以使用 \verb|\[  \]| 包裹,例如
\[
E=mc^2
\]
其中 $ E $ 是能量,$ m $ 是质量,$ c $ 是光速。

如果希望某个公式带编号,并且在后文中引用可以参考下面的写法:
\begin{equation}
	E=mc^2
	\label{eq:energy}
\end{equation}
式\cref{eq:energy}是质能方程。

多行公式有时候希望能够在特定的位置对齐,以下是其中一种处理方法。
\begin{align}
	P & = UI \\
	& = I^2R
\end{align}
\verb|&| 是对齐的位置, \verb|&| 可以有多个,但是每行的个数要相同。

矩阵的输入也不难。
\[
\mathbf{X} = \left(
\begin{array}{cccc}
	x_{11} & x_{12} & \ldots & x_{1n}\\
	x_{21} & x_{22} & \ldots & x_{2n}\\
	\vdots & \vdots & \ddots & \vdots\\
	x_{n1} & x_{n2} & \ldots & x_{nn}\\
\end{array} \right)
\]

分段函数这些可以用 \verb|case| 环境,但是它要放在数学环境里面。
\[
f(x) =
\begin{cases}
	0 &  x \text{为无理数} ,\\
	1 &  x \text{为有理数} .
\end{cases}
\]
在数学环境里面,字体用的是数学字体,一般与正文字体不同。假如要公式里面有个别文字,则需要把这部分放在 \verb|text| 环境里面,即 \verb|\text{文本环境}| 。

公式中个别需要加粗的字母可以用 \verb|$\bm{math symbol}$| 。如 $ \alpha a\bm{\alpha a} $ 。

以上仅简单介绍了基础的使用,对于更复杂的需求,可以阅读相关的宏包手册,如 \href{http://texdoc.net/texmf-dist/doc/latex/amsmath/amsldoc.pdf}{amsmath}。

希腊字母这些如果不熟悉,可以去查找符号文件 \href{http://mirrors.ctan.org/info/symbols/comprehensive/symbols-a4.pdf}{symbols-a4.pdf} ,也可以去 \href{http://detexify.kirelabs.org/classify.html}{detexify} 网站手写识别。另外还有数学公式识别软件 \href{https://mathpix.com/}{mathpix} 。
无序列表是这样的:
\begin{itemize}
	\item one
	\item two
	\item ...
\end{itemize}

有序列表是这样子的:
\begin{enumerate}
	\item one
	\item two
	\item ...
\end{enumerate}

\begin{definition}
	定义环境
	\label{def:nosense}
\end{definition}
\cref{def:nosense}除了告诉你怎么使用这个环境以外,没有什么其它的意义。

除了 definition 环境,还可以使用 theorem 、lemma、corollary、assumption、conjecture、axiom、principle、problem、example、proof、solution 这些环境,根据论文的实际需求合理使用。

\begin{theorem}
	这是一个定理。
	\label{thm:example}
\end{theorem}
由\cref{thm:example}我们知道了定理环境的使用。

\begin{lemma}
	这是一个引理。
	\label{lem:example}
\end{lemma}
由\cref{lem:example}我们知道了引理环境的使用。

\begin{corollary}
	这是一个推论。
	\label{cor:example}
\end{corollary}
由\cref{cor:example}我们知道了推论环境的使用。

\begin{assumption}
	这是一个假设。
	\label{asu:example}
\end{assumption}
由\cref{asu:example}我们知道了假设环境的使用。

\begin{conjecture}
	这是一个猜想。
	\label{con:example}
\end{conjecture}
由\cref{con:example}我们知道了猜想环境的使用。

\begin{axiom}
	这是一个公理。
	\label{axi:example}
\end{axiom}
由\cref{axi:example}我们知道了公理环境的使用。

\begin{principle}
	这是一个定律。
	\label{pri:example}
\end{principle}
由\cref{pri:example}我们知道了定律环境的使用。

\begin{problem}
	这是一个问题。
	\label{pro:example}
\end{problem}
由\cref{pro:example}我们知道了问题环境的使用。

\begin{example}
	这是一个例子。
	\label{exa:example}
\end{example}
由\cref{exa:example}我们知道了例子环境的使用。

\begin{proof}
	这是一个证明。
	\label{prf:example}
\end{proof}
由\cref{prf:example}我们知道了证明环境的使用。

\begin{solution}
	这是一个解。
	\label{sol:example}
\end{solution}
由\cref{sol:example}我们知道了解环境的使用。